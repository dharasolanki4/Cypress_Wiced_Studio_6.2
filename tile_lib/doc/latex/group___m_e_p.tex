\hypertarget{group___m_e_p}{}\section{Multi-\/endpoint T\+OA}
\label{group___m_e_p}\index{Multi-\/endpoint T\+OA@{Multi-\/endpoint T\+OA}}
Multi-\/endpoint T\+OA (M\+EP) is a means by which multiple apps from the same phone may communicate with a single Tile device simultaneously. M\+EP establishes the concept of T\+OA {\itshape channels}, which are T\+OA data streams that can be accessed concurrently. A channel is specified by prepending the channel\textquotesingle{}s C\+ID to the T\+OA message\+:

\tabulinesep=1mm
\begin{longtabu} spread 0pt [c]{*{2}{|X[-1]}|}
\hline
\rowcolor{\tableheadbgcolor}\textbf{ C\+ID  }&\textbf{ T\+OA message   }\\\cline{1-2}
\endfirsthead
\hline
\endfoot
\hline
\rowcolor{\tableheadbgcolor}\textbf{ C\+ID  }&\textbf{ T\+OA message   }\\\cline{1-2}
\endhead
1 byte  &A\+T\+T\+\_\+\+M\+TU -\/ 3 -\/ 1 bytes (typically 19 bytes)   \\\cline{1-2}
\end{longtabu}


\subsection*{Channel types}

There are three types of channel\+:


\begin{DoxyItemize}
\item The connectionless channel\+: a special channel for unauthenticated traffic.
\item The broadcast channel\+: a special channel for one-\/way communication from the T\+OA server to all T\+OA clients which are listening for broadcasts.
\item Authenticated channels\+: channels that are allocated to a T\+OA client and require a message integrity check (M\+IC) to be appended the T\+OA message.
\end{DoxyItemize}

\subsubsection*{Connectionless channel}

Transactions on the connectionless channel have the following format\+:

\tabulinesep=1mm
\begin{longtabu} spread 0pt [c]{*{4}{|X[-1]}|}
\hline
\rowcolor{\tableheadbgcolor}\textbf{ T\+O\+A\+\_\+\+C\+O\+N\+N\+E\+C\+T\+I\+O\+N\+L\+E\+S\+S\+\_\+\+C\+ID  }&\textbf{ Token  }&\textbf{ T\+OA code  }&\textbf{ T\+OA Pay   }\\\cline{1-4}
\endfirsthead
\hline
\endfoot
\hline
\rowcolor{\tableheadbgcolor}\textbf{ T\+O\+A\+\_\+\+C\+O\+N\+N\+E\+C\+T\+I\+O\+N\+L\+E\+S\+S\+\_\+\+C\+ID  }&\textbf{ Token  }&\textbf{ T\+OA code  }&\textbf{ T\+OA Pay   }\\\cline{1-4}
\endhead
1 byte  &4 bytes  &1 byte  &Up to T\+O\+A\+\_\+\+M\+PS bytes   \\\cline{1-4}
\end{longtabu}


The token is a value randomly generated by the T\+OA client in order to identify the transaction. When the T\+OA server receives a command on the connectionless channel, it will send a response with the same token as in the command. Example\+:


\begin{DoxyImageNoCaption}
  \mbox{\includegraphics[width=\textwidth,height=\textheight/2,keepaspectratio=true]{inline_mscgraph_3}}
\end{DoxyImageNoCaption}


\subsubsection*{Authenticated channels}

Transactions on authenticated channels have the following format\+:

\tabulinesep=1mm
\begin{longtabu} spread 0pt [c]{*{4}{|X[-1]}|}
\hline
\rowcolor{\tableheadbgcolor}\textbf{ C\+ID  }&\textbf{ T\+OA code  }&\textbf{ T\+OA payload  }&\textbf{ M\+IC   }\\\cline{1-4}
\endfirsthead
\hline
\endfoot
\hline
\rowcolor{\tableheadbgcolor}\textbf{ C\+ID  }&\textbf{ T\+OA code  }&\textbf{ T\+OA payload  }&\textbf{ M\+IC   }\\\cline{1-4}
\endhead
1 byte  &1 byte  &Up to T\+O\+A\+\_\+\+M\+PS bytes  &4 bytes   \\\cline{1-4}
\end{longtabu}


\paragraph*{Opening an authenticated channel}

To use an authenticated channel, a T\+OA client must use T\+O\+A\+\_\+\+C\+M\+D\+\_\+\+O\+P\+E\+N\+\_\+\+C\+H\+A\+N\+N\+EL on the connectionless channel, generate a session key, and then send the T\+O\+A\+\_\+\+C\+M\+D\+\_\+\+R\+E\+A\+DY on the newly allocated channel.


\begin{DoxyImageNoCaption}
  \mbox{\includegraphics[width=\textwidth,height=\textheight/2,keepaspectratio=true]{inline_mscgraph_4}}
\end{DoxyImageNoCaption}


(See documentation for each command/response for format of each message).

\paragraph*{Session key generation}

The session key is generated using H\+M\+A\+C-\/\+S\+H\+A256 with the Tile\textquotesingle{}s authentication key as the key and the message as

\tabulinesep=1mm
\begin{longtabu} spread 0pt [c]{*{4}{|X[-1]}|}
\hline
\rowcolor{\tableheadbgcolor}\textbf{ RandA  }&\textbf{ RandT  }&\textbf{ C\+ID  }&\textbf{ Token from T\+O\+A\+\_\+\+C\+M\+D\+\_\+\+O\+P\+E\+N\+\_\+\+C\+H\+A\+N\+N\+EL   }\\\cline{1-4}
\endfirsthead
\hline
\endfoot
\hline
\rowcolor{\tableheadbgcolor}\textbf{ RandA  }&\textbf{ RandT  }&\textbf{ C\+ID  }&\textbf{ Token from T\+O\+A\+\_\+\+C\+M\+D\+\_\+\+O\+P\+E\+N\+\_\+\+C\+H\+A\+N\+N\+EL   }\\\cline{1-4}
\endhead
14 bytes  &13 bytes  &1 byte  &4 bytes   \\\cline{1-4}
\end{longtabu}


The session key is the first 16 bytes of the hash.

\paragraph*{M\+IC generation}

The M\+IC for authenticated channels is generated using H\+M\+A\+C-\/\+S\+H\+A256 with the session key as the key. The message to be hashed varies depending on whether the packet originated with the T\+OA client or server. For a packet sent from the client to the server, the message is

\tabulinesep=1mm
\begin{longtabu} spread 0pt [c]{*{5}{|X[-1]}|}
\hline
\rowcolor{\tableheadbgcolor}\textbf{ Nonce A  }&\textbf{ 1  }&\textbf{ Text length  }&\textbf{ Text  }&\textbf{ 0 --   }\\\cline{1-5}
\endfirsthead
\hline
\endfoot
\hline
\rowcolor{\tableheadbgcolor}\textbf{ Nonce A  }&\textbf{ 1  }&\textbf{ Text length  }&\textbf{ Text  }&\textbf{ 0 --   }\\\cline{1-5}
\endhead
8 bytes  &1 byte  &1 byte  &Text length bytes  &Until message is 32 bytes long   \\\cline{1-5}
\end{longtabu}


and a packet from the server to the client has the message format

\tabulinesep=1mm
\begin{longtabu} spread 0pt [c]{*{5}{|X[-1]}|}
\hline
\rowcolor{\tableheadbgcolor}\textbf{ Nonce T  }&\textbf{ 0  }&\textbf{ Text length  }&\textbf{ Text  }&\textbf{ 0 --   }\\\cline{1-5}
\endfirsthead
\hline
\endfoot
\hline
\rowcolor{\tableheadbgcolor}\textbf{ Nonce T  }&\textbf{ 0  }&\textbf{ Text length  }&\textbf{ Text  }&\textbf{ 0 --   }\\\cline{1-5}
\endhead
8 bytes  &1 byte  &1 byte  &Text length bytes  &Until message is 32 bytes long   \\\cline{1-5}
\end{longtabu}


The M\+IC is the first four bytes of the hash. The nonce values are simply counters which increment before each M\+IC is generated. The text is the T\+OA message, without the C\+ID or the M\+IC. Example C code for generating the M\+IC of a message from the T\+OA client to the T\+OA server\+:


\begin{DoxyCode}
\textcolor{keywordtype}{void} mic\_example(\textcolor{keywordtype}{void})
\{
    \textcolor{comment}{// Full payload}
    uint8\_t msg[] = \{
        cid,
        TOA\_CMD\_SONG,
        TOA\_SONG\_CMD\_PLAY,
        TILE\_SONG\_FIND,
        3,               \textcolor{comment}{// Strength 3}
        0, 0, 0, 0       \textcolor{comment}{// Space for MIC}
    \};

    uint8\_t hash\_message[32] = \{0\};

    nonceA++;

    memcpy(hash\_message, &nonceA, 8);      \textcolor{comment}{// nonceA since packet is from client to server}
    hash\_message[8] = 1;                   \textcolor{comment}{// 1 since direction is from client to server}
    hash\_message[9] = \textcolor{keyword}{sizeof}(msg) - 1 - 4; \textcolor{comment}{// -1 for CID and -4 for MIC}
    memcpy(hash\_message + 10, msg + 1, hash\_message[9]);

    \textcolor{comment}{// auth\_key is 16 bytes, message is 32 bytes, and desired output is 4 bytes}
    hmac\_sha256(auth\_key, 16, hash\_message, 32, &msg[\textcolor{keyword}{sizeof}(msg)-4], 4);
\}
\end{DoxyCode}


\paragraph*{Closing a channel}

T\+OA clients and servers may close a channel at any time using T\+O\+A\+\_\+\+C\+M\+D\+\_\+\+C\+L\+O\+S\+E\+\_\+\+C\+H\+A\+N\+N\+EL and T\+O\+A\+\_\+\+R\+S\+P\+\_\+\+C\+L\+O\+S\+E\+\_\+\+C\+H\+A\+N\+N\+EL, respecitvely. Format for a close channel message\+:

\tabulinesep=1mm
\begin{longtabu} spread 0pt [c]{*{5}{|X[-1]}|}
\hline
\rowcolor{\tableheadbgcolor}\textbf{ C\+ID  }&\textbf{ T\+O\+A\+\_\+\+C\+M\+D/\+R\+S\+P\+\_\+\+C\+L\+O\+S\+E\+\_\+\+C\+H\+A\+N\+N\+EL  }&\textbf{ T\+O\+A\+\_\+\+C\+H\+A\+N\+N\+E\+L\+\_\+\+C\+L\+O\+S\+E\+\_\+\+R\+E\+A\+S\+O\+NS code  }&\textbf{ Payload  }&\textbf{ M\+IC   }\\\cline{1-5}
\endfirsthead
\hline
\endfoot
\hline
\rowcolor{\tableheadbgcolor}\textbf{ C\+ID  }&\textbf{ T\+O\+A\+\_\+\+C\+M\+D/\+R\+S\+P\+\_\+\+C\+L\+O\+S\+E\+\_\+\+C\+H\+A\+N\+N\+EL  }&\textbf{ T\+O\+A\+\_\+\+C\+H\+A\+N\+N\+E\+L\+\_\+\+C\+L\+O\+S\+E\+\_\+\+R\+E\+A\+S\+O\+NS code  }&\textbf{ Payload  }&\textbf{ M\+IC   }\\\cline{1-5}
\endhead
1 byte  &1 byte  &1 byte  &Varies  &4 bytes   \\\cline{1-5}
\end{longtabu}


There are certain behaviors expected of a close channel request.


\begin{DoxyEnumerate}
\item The sender and recipient of a close channel command/response must immediately close the channel and not send any further data over the channel.
\item The close channel command/response must contain a reason code and the payload must conform to the format specified for that particular reason code.
\item The close channel response may be sent over the broadcast C\+ID, and in this case, all clients must close their open channels.
\end{DoxyEnumerate}

\subsubsection*{Broadcast channel}

Transactions on the broadcast channel have the following format\+:

\tabulinesep=1mm
\begin{longtabu} spread 0pt [c]{*{4}{|X[-1]}|}
\hline
\rowcolor{\tableheadbgcolor}\textbf{ T\+O\+A\+\_\+\+B\+R\+O\+A\+D\+C\+A\+S\+T\+\_\+\+C\+ID  }&\textbf{ T\+OA code  }&\textbf{ T\+OA payload  }&\textbf{ M\+IC   }\\\cline{1-4}
\endfirsthead
\hline
\endfoot
\hline
\rowcolor{\tableheadbgcolor}\textbf{ T\+O\+A\+\_\+\+B\+R\+O\+A\+D\+C\+A\+S\+T\+\_\+\+C\+ID  }&\textbf{ T\+OA code  }&\textbf{ T\+OA payload  }&\textbf{ M\+IC   }\\\cline{1-4}
\endhead
1 byte  &1 byte  &Up to T\+O\+A\+\_\+\+M\+PS bytes  &4 bytes   \\\cline{1-4}
\end{longtabu}


\paragraph*{Broadcast key generation}

The broadcast key is generated using H\+M\+A\+C-\/\+S\+H\+A256 with the Tile\textquotesingle{}s authentication key as the key and the message as

\tabulinesep=1mm
\begin{longtabu} spread 0pt [c]{*{3}{|X[-1]}|}
\hline
\rowcolor{\tableheadbgcolor}\textbf{ RandT\mbox{[}0\+:9\mbox{]}  }&\textbf{ Tile ID  }&\textbf{ 0 -\/---   }\\\cline{1-3}
\endfirsthead
\hline
\endfoot
\hline
\rowcolor{\tableheadbgcolor}\textbf{ RandT\mbox{[}0\+:9\mbox{]}  }&\textbf{ Tile ID  }&\textbf{ 0 -\/---   }\\\cline{1-3}
\endhead
10 bytes  &8 bytes  &14 bytes   \\\cline{1-3}
\end{longtabu}


where RandT\mbox{[}0\+:9\mbox{]} is the first 10 bytes of the RandT value received in the T\+O\+A\+\_\+\+R\+S\+P\+\_\+\+O\+P\+E\+N\+\_\+\+C\+H\+A\+N\+N\+EL. The broadcast key is the first 16 bytes of the hash.

\paragraph*{Broadcast M\+IC generation}

The M\+IC for the broadcast channel is generated using H\+M\+A\+C-\/\+S\+H\+A256 with the broadcast key as the key. The message to be hashed is

\tabulinesep=1mm
\begin{longtabu} spread 0pt [c]{*{4}{|X[-1]}|}
\hline
\rowcolor{\tableheadbgcolor}\textbf{ Nonce B  }&\textbf{ Text length  }&\textbf{ Text  }&\textbf{ 0 --   }\\\cline{1-4}
\endfirsthead
\hline
\endfoot
\hline
\rowcolor{\tableheadbgcolor}\textbf{ Nonce B  }&\textbf{ Text length  }&\textbf{ Text  }&\textbf{ 0 --   }\\\cline{1-4}
\endhead
8 bytes  &1 byte  &Text length bytes  &Until message is 32 bytes long   \\\cline{1-4}
\end{longtabu}


The M\+IC is the first four bytes of the hash. The nonce value is simply a counter which increments before each M\+IC is generated. The T\+O\+A\+\_\+\+R\+S\+P\+\_\+\+R\+E\+A\+DY message includes the current value of NonceB. 